\documentclass[a4paper,12pt]{article}

\usepackage{cmap}					
\usepackage[T2A]{fontenc}
\usepackage[utf8]{inputenc}
\usepackage[english,russian]{babel}
\usepackage{hyperref}
\usepackage{graphicx}

\hypersetup{
    colorlinks=true,
    linkcolor=blue,
    filecolor=magenta,      
    urlcolor=cyan,
}

\usepackage{amsmath,amsfonts,amssymb,mathtools}
\usepackage{icomma}

\usepackage{euscript}
\usepackage{mathrsfs}
\usepackage{graphicx}
\usepackage{eso-pic}

\begin{document}

\begin{titlepage}

\thispagestyle{empty}

\centerline{НИУ ИТМО}
\centerline{Факультет Информационных Технологий и Программирования}
\centerline{Направление "Прикладная Математика и Информатика"}

\vfill

\centerline{\huge{Лабораторная работа 3}}
\centerline{\large{курса ``Методы Оптимизации'' }}
\vspace{1cm} 
\centerline{\large{Выполнили Раков Николай, Булкина Милена}}
\vfill

\centerline{Санкт-Петербург, 2021}
\clearpage
\end{titlepage}

%\title{Методы Оптимизации. Лабораторная работа №3}
%\author{Раков Николай, Булкина Милена}
%\date{}

\section{Постановка задания}
\begin{enumerate}
\item Реализовать прямой метод решения СЛАУ на основе LU-разложения с учетом следующих требований:
\begin{itemize}
\item формат матрицы – профильный;
\item размерность матрицы, элементы матрицы и вектор правой части читать из файлов, результаты записывать в файл;
\item в программе резервировать объём памяти, необходимый для хранения в нем только одной матрицы и необходимого числа векторов (то есть треугольные матрицы, полученные в результате разложения, должны храниться на месте исходной матрицы);
\item элементы матрицы обрабатывать в порядке, соответствующем формату хранения, то есть необходимо работать именно со столбцами верхнего и строками нижнего треугольников.
\end{itemize}

\item Провести исследование реализованного метода на матрицах, число
обусловленности которых регулируется за счёт изменения диагонального преобладания (то есть оценить влияние увеличения числа обусловленности на точность решения).
\begin{itemize}
\item Оценить погрешность решения для каждого k, для которого система вычислительно разрешима.
\item Для одного из значений k попытаться найти операцию, вызывающую скачкообразное накопление погрешности, пояснить полученные результаты.
\end{itemize}

\item Провести аналогичные исследования на матрицах Гильберта различной размерности. Матрица Гильберта размерности k строится следующим образом:
\begin{equation*}
    a_{ij} = \frac{1}{i + j - 1}, i,j = \overline{1,k}
\end{equation*}

\item Реализовать метод Гаусса с выбором ведущего элемента для плотных матриц. Сравнить метод Гаусса по точности получаемого решения и по количеству действий с реализованным прямым методом LU – разложения.

\item Реализовать метод сопряженных градиентов для решения СЛАУ, матрица которых хранится в разреженном строчно – столбцовом и является симметричной.

\end{enumerate}

\section{Формат хранения матриц}
\subsection{Плотный}
Плотный формат хранения матриц используется, когда матрица не обладает определенной структурой и имеет малые размеры. Этот формат требует $\mathcal{O}(n^2)$ памяти.

\subsection{Профильный}
Профильный формат хранения матриц используется, когда ненулевые элементы матрицы расположены в произвольном порядке, но при этом они сосредоточены у главной диагонали. 
Этот формат занимает меньше памяти, чем плотный, если ненулевые элементы плотно расположены около главной диагонали.

\noindentМассивы, необходимые для хранения профильной квадратной матрицы:
\begin{itemize}
    \item Массивы al и au - вещественные массивы, которые хранят внедиагональные элементы нижнего (по строкам) и верхнего (по столбцам) треугольника матрицы.
    \item Массив ia[n] хранит информацию о профиле. Он содержит указатели начала строк (столбцов) нижнего (верхнего) треугольника в массиве al (au). Индексный массив ia формируется следующим образом:
\begin{itemize}
    \item первые два элемента индексного массива ia всегда равны 1: ia[1] = ia[2] = 1, так как в первой строке в нижнем треугольнике нет внедиагональных элементов
    \item к элементу ia[k] добавляется количество элементов в профиле k -ой строки и получается элемент ia[k + 1].
\end{itemize}
    \item di[n] - вещественный массив. Он хранит все диагональные элементы.
\end{itemize}

\subsection{Разреженный строчно-столбцовый симметричный формат}
В отличие от профильного формата дополнительно необходим еще один массив, содержащий информацию о положении внедиагонального элемента в строке (для нижнего треугольника) или в столбце (для верхнего треугольника). Используется, когда количество нулевых элементов велико.

\noindent Необходимые массивы для данного формата:
\begin{itemize}
    \item Вещественные массивы al и au, которые хранят внедиагональные элементы нижнего (по строкам) и верхнего (по столбцам) треугольника матрицы.
    \item Целочисленный массив ja хранит номера столбцов (строк) внедиагональных элементов нижнего (верхнего) треугольника матрицы.
    \item Целочисленный массив ia[n] хранит указатели начала строк (столбцов) в массивах ja , al и au.
\end{itemize}

\section{Решение СЛАУ}
\begin{equation*} 
 \begin{cases}
   a_{11}x_1 + a_{12}x_2 \ldots +  a_{1n}x_n = b_1\\ 
   a_{21}x_1 + a_{22}x_2 \ldots +  a_{2n}x_n = b_2\\ 
   \ldots\\
   a_{n1}x_1 + a_{n2}x_2 \ldots +  a_{nn}x_n = b_n
 \end{cases} 
\end{equation*}
\subsection{Метод Гаусса с выбором главного элемента}
Данный метод имеет преимущество перед обычным методом последовательного исключения Гаусса, так как в последнем среди ведущих элементов могут оказаться очень маленькие по абсолютной величине. Тогда при делении на данные элементы получается большая вычислительная погрешность.

В данном методе выбирается наибольший по модулю элемент.

\subsection{LU -  разложение}


\end{document}
